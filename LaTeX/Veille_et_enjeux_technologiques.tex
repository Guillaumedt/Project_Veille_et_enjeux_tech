\documentclass{ieeeaccess}
\usepackage{cite}
\usepackage{amsmath,amssymb,amsfonts}
\usepackage{algorithmic}
\usepackage{graphicx}
\usepackage{textcomp}
\def\BibTeX{{\rm B\kern-.05em{\sc i\kern-.025em b}\kern-.08em
    T\kern-.1667em\lower.7ex\hbox{E}\kern-.125emX}}
\begin{document}
\title{Veille \& enjeux technologiques}
\author{
\uppercase{Sujet 2 : Les métriques d'évaluation de la qualité de l'image}
\newline
\newline
\uppercase{Lucas Artaud}\authorrefmark{1},
\uppercase{Guillaume de Trentinian}\authorrefmark{2}, 
\uppercase{Maryam Dollet}\authorrefmark{3},
\uppercase{Iswarya Sivasubramaniam}\authorrefmark{4}
}
\address[1]{ESILV \& EMLV, 92400 Courbevoie, France 
(e-mail: lucas.artaud@edu.devinci.fr)}
\address[2]{ESILV, 92400 Courbevoie, France
(e-mail: guillaume.de\_trentinian@edu.devinci.fr)}
\address[3]{ESILV \& EMLV, 92400 Courbevoie, France
(e-mail: maryam.dollet@edu.devinci.fr)}
\address[4]{ESILV \& EMLV, 92400 Courbevoie, France
(e-mail: iswarya.sivasubramaniam@edu.devinci.fr)}
\tfootnote{Ce travail a été supporté en partie par l'ESILV.}

\begin{abstract}
La qualité d’image est un concept subjectif qui reflète la perception d’un observateur sur l’adéquation d’une image pour un usage spécifique. C’est un domaine crucial dans de nombreux secteurs, notamment la photographie, la cinématographie, la télévision, et plus récemment, les médias numériques.
Dans un monde où la visualisation est cruciale, l'évaluation de la qualité des images est devenue essentielle. En fournissant des indicateurs objectifs sur la performance des algorithmes de traitement d'image, les métriques dédiées à cette évaluation jouent un rôle essentiel. Ces mesures vont au-delà de la simple perception visuelle, englobant des aspects tels que la netteté, la fidélité des couleurs et la préservation des détails, offrant ainsi un éclairage essentiel pour améliorer les technologies visuelles.
\end{abstract}

\begin{keywords}
intelligence artificielle, image, métriques d'évaluation
\end{keywords}

\titlepgskip=-15pt

\maketitle

\section{Rappel de la problématique}
\PARstart{C}{omment} évaluer la qualité des images produites par les modèles génératifs de création d’images ?
Nous explorerons les critères permettant de juger la fidélité perceptive et la capacité à évaluer des images déformées. Notre approche consistera à maximiser différentes métriques présentes dans la littérature afin de déterminer si une image générée est de haute qualité. L'objectif ultime est de développer un modèle génératif qui oriente la génération d'images vers une amélioration constante de leur qualité.

\section{Synthèse des papiers retenus}

\subsection{Information sur le papier et en quoi ça répond à la problématique}

abc

\subsection{Explication du fonctionnement des algos (papier par papier)}

abc

\subsection{Critiques des algos (avantages/inconvénients)}

abc

\subsection{Résultats expérimentaux : Description des datasets utilisés et résultats obtenus}

abc

\section{Justification du choix de l'article}

\subsection{Les articles utilisés et pourquoi (MANIQA et BRISQUE)}

abc

\subsection{Explication du fonctionnement (notre expérimentation)}

abc

\subsection{Les résultats de l'expérimentation}

abc

\section{Bilan personnel}

abc

\begin{thebibliography}{00}

\bibitem{b1}
T. Karras, S. Laine, M. Aittala, J. Hellsten, J. Lehtinen, \& T. Aila (2020).
\emph{Analyzing and Improving the Image Quality of StyleGAN} [Online].
\newline
Available:
\underline{https://arxiv.org/abs/1912.04958}.

\bibitem{b2}
A. Mittal, Anush K. Moorthy \& A. Bovik
\emph{No-Reference Image Quality Assessment in the Spatial Domain} [Online].
\newline
Available:
\underline{https://www.semanticscholar.org/paper/No-Reference-Image-Quality-Assessment-in-the-Domain-Mittal-Moorthy/a2cad4e4fd946adf6cc87e483b2ba18579de1264}

\bibitem{b3}
G. Zhai \& X. Min
\emph{Perceptual image quality assessment: a survey} [Online].
\newline
Available:
\underline{https://www.researchgate.net/publication/341011181\textunderscore Perceptual\textunderscore image\textunderscore quality\textunderscore assessment\textunderscore a\textunderscore survey}

\bibitem{b4}
S. Yang, T. Wu, S. Shi, S. Lao, Y. Gong, M. Cao, J. Wang \& Y. Yang
\emph{MANIQA: Multi-dimension Attention Network for No-Reference Image Quality Assessment} [Online].
\newline
Available:
\underline{http://press-pubs.uchicago.edu/founders/}

\end{thebibliography}

\EOD

\end{document}
